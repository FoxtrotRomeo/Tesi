\chapter{Stimolazione}
\paragraph{Simons }\label{par:sim}\cite{Simons2017} The Use of Brain 
Stimulation in Dysphagia 
Management.
Si spiega la differenza tra i vari tipi di stimolazione, come stimolazione 
magnetica transcranica e stimolazione elettrica transcranica diretta.
Si evidenzia come studi diversi, nel tempo, abbiano evidenziato diverse 
efficacie di questi due metodi, e come anche la sede di applicazione (emisfero 
ispi- o contro-laterale), l'intensità della stimolazione, la durata delle 
sedute e il numero delle sedute influenzino i risultati.

Evidenzia, infine, come si siano incontrate differenze significative rispetto 
alla stimolazione sham con la stimolazione vera e propria, ma anche come solo 
la stimolazione magnetica crei delle vere differenze.
Si auspicano successivi studi di qualità per indagare meglio gli effetti, in 
quanto a causa delle ridotte proporzioni dei campioni si va incontro ad errori 
di tipo 1.

\paragraph{Crary}\cite{Crary2014} parla anche\ref{par:cra} di 
elettrostimolazione, e in 
particolare riporta come molti sforzi nella ricerca recente si stiano 
concentrando su questo metodo.
Restano molte possibili variabili: il posizionamento degli elettrodi, il tempo 
e l'intensità della stimolazione.
A seconda di come vengono manipolate queste variabili, cambia anche il 
risultato delle sperimentazioni, potendo arrivare anche a risultati 
contrapposti.

Questa metodica mostra dei risultati discreti e promettenti, soprattutto quando 
viene proposta a tipologie di pazienti specifiche.
\'E quindi importante, nel futuro, che la ricerca si concentri sul rigore 
scientifico da dare a queste sperimentazioni, e che si riescano a distinguere 
le varie variabili.
Per ora, è bene che la stimolazione elettrica trancutanea resti un'aggiunta 
alle terapie basate su esercizi, e non le sostituisca completamente, e che 
venga applicata solo nei casi in cui la letteratura dà indicazioni specifiche a 
riguardo.

\paragraph{Liao}\label{par:lia}\cite{Liao2016} parla della stimolazione 
magnetica transcranica dopo lo stroke.
Divide gli studi che analizza in base a dove applicano la stimolazione e alla 
frequenza.

I risultati dicono che la stimolazione ad alta frequenza ($>1MHz$) è più 
efficace rispetto a quella a bassa frequenza.
Inoltre, le stimolazioni bilaterali, o dell'emisfero sano, sono più efficaci 
rispetto a quelle dell'emisfero colpito.

Si è trovato, inoltre, che la stimolazione ad alta frequenza è eccitatoria, 
mentre quella a bassa frequenza è inibitoria.
Dato che la deglutizione è rappresentata nella corteccia cerebrale 
bilateralmente, si pensa che l'effetto benefico della stimolazione sia dovuto 
al bilanciamento recuperato tra i due emisferi, quindi che si ottenga inibendo 
l'emisfero sano o stimolando quello colpito.
Essendo la rappresentazione bilaterale, si è trovato che c'è un emisfero 
dominante, indipendente dall'essere destri o mancini, e che chi viene colpito 
nell'emisfero dominante ha più probabilità di sviluppare disfagia dopo lo 
stroke, e un recupero meno buono.
Il meccanismo che porta al miglioramento, comunque, non è ancora chiaro.

Gli effetti di una terapia breve (1 o 2 settimane) durano anche al follow up 
dopo 4 settimane.
Questo tipo di terpia dunque è efficace, sicura, e permette un miglioramento 
permanente del paziente.

\paragraph{Robbins}\cite{Robbins2013} parla anche\ref{par:rob2013} della 
stimolazione elettrica e magnetica transcranica.
Affermano che gli studi sono promettenti per il futuro, perché sono 
qualitativamente molto validi e mostrano dei buoni risultati.
L'elettrostimolazione invece viene etichettata come avente studi di qualità 
inferiore, a volte senza controlli o con numeri troppo ristretti per mostrare 
risultati significativi.

\paragraph{Wirth}\ref{par:wir} parla anche dei metodi tecnologici per trattare 
la disfagia, come la stimolazione elettrica e quella magnetica.

Vengono riportati degli studi sulla stimolazione periferica, a mezzo correnti 
(miglioramento dei parametri della deglutizione, diminuzione del tempo di 
degenza), gusti (acido (Retrazione della lingua, elevazione del palato, 
apertura dello sfintere esofageo) e piccante (tempi e meccanismi protettivi 
migliorati, effetti a lungo termine non testati) e temperature (in particolare 
il freddo (miglioramento della sensibilità buccale e del timing di 
attivazione)).
Questi studi spesso hanno pochi partecipanti, ma sembra che riescano ad agire 
sul meccanismo della deglutizione, anche se studi di qualità devono seguire per 
confermare o negare i risultati.

La stimolazione magnetica è stata testata in pazienti con stroke, ma non in 
pazienti geriatrici.
Sembra che sia efficace, sia con stimolazioni nell'emisfero affetto che in 
quello sano, e sia con alte che con basse frequenze (meglio rifarsi a studi più 
dettagliati).

La stimolazione elettrica transcranica è stata usata in pazienti con stroke, ma 
non in pazienti geriatrici.
Anche in questo caso, gli studi non sono concordi nel metodo di applicazione 
(quale emisfero, con quale potenza e con quale frequenza), e quindi c'è bisogno 
di ulteriori studi più specifici per poter trarre conclusioni decisive.

La nutrizione alternativa via PEG è da preferire per i minori rischi e i 
maggiori risultati se il tempo di applicazione supera le due settimane, mentre 
per tempi più corti il sondino nasogastrico sembra avere outcome migliori.
In entrambi i casi, dovrebbe essere mantenuto un minimo income orale, 
compatibilmente con le esigenze del paziente, e dovrebbe essere iniziata la 
riabilitazione della deglutizione il prima possibile.

Nelle demenze, diverse demenze portano a diversi problemi con la deglutizione.
Nei pazienti con demenza vascolare, è consigliato l'uso di esercizi motori, 
perché ci sono problemi nella formazione e progressione del bolo.
In questi pazienti, la PEG non riduce l'aspirazione, che viene dalle secrezioni.

Nel Parkinson, è consigliato un approccio individualizzato sia per la diagnosi 
che per il trattamento, proponendo comunque i liquidi addensati, che, 
compatibilmente con le preferenze del paziente, sembrano essere più efficaci di 
alcune manovre nel ridurre le aspirazioni.
Una nuova terapia è quella di proporre un biofeedback tramite la 
laringofaringoscopia.
\'E particolarmente importante anche tenere sotto controllo i livelli di L-DOPA 
in questi pazienti, durante la terapia e durante i pasti.

\paragraph{Macrae}\label{par:mac} \cite{Macrae2014} parla di come si usino la 
stimolazione magnetica transcranica e i potenziali muscolari evocati per 
indagare la neuroplasticità indotta dai vari trattamenti.
In particolare, analizza tutte le variabili della stimolazione, e come queste 
variabili influenzino i risultati in eccitabilità delle varie zone cerebrali, 
fornendo una guida per ridurre, o almeno per essere coscienti di queste 
variabilità.

\paragraph{Doeltghen}\label{par:doe} \cite{Doeltgen2015} effettua una revisione 
degli studi che utilizzano la stimolazione transcranica come metodo di 
trattamento per la disfagia dopo l'ictus.

Divide la stimolazione in stimolazione elettrica, magnetica e accoppiata 
tramagnetica ed elettrica.

Nella stimolazione elettrica, vengono presi in considerazione studi su soggetti 
sani, sani a cui viene provocata una lesione virtuale con la stimolazione 
magnetica, e colpiti da ictus.
In tutti i soggetti, la stimolazione elettrica anodale migliora l'eccitabilità 
della corteccia, e nei soggetti con ictus migliorano le prove deglutitorie, in 
maniera significativamente diversa tra i pazienti che ricevono la stimolazione 
e i gruppi sham.
Viene indicato che se presentata assieme ad un trattamento tradizionale, la 
stimolazione elettrica può dare una finestra di maggiore eccitabilità, in cui 
il trattamento potrebbe essere più efficace.
Si evidenzia la necessità di indagare meglio l'effetto dell'intensità di 
corrente applicata.

Nella stimolazione magnetica, si è trovato che i pazienti che andavano incontro 
a stimolazione, sia nell'emisfero sano, sia in quello colpito, sia in entrambi, 
avevano miglioramenti significativamente migliori di quelli del gruppo sham.
Nei soggetti sani, si è trovato che l'inibizione di un emisfero ha migliorato i 
tempi di deglutizione, il che è strano perché questo tipo di intervento viene 
usato per simulare lesioni.
La variabilità dei protocolli richiede ulteriore specificità nella 
somministrazione della stimolazione, e richiede studi comparativi in casistiche 
di pazienti con lesioni simili, per permettere di comprendere l'effetto della 
stimolazione ed inserirla nella pratica clinica.

Nella stimolazione accoppiata, si sono trovati risultati migliori stimolando 
sensitivamente la faringe mentre si stimolava la corteccia, che stimolando solo 
la corteccia.
Questi risultati sono stati migliori sia per quanto riguarda l'eccitabilità 
corticale, sia per quanto riguarda misure della disfagia e i tempi di transito.
Risulta quindi che l'accoppiamento della stimolazione centrale e periferica va 
indagato a fondo perché risulta potenzialmente efficace.

In conclusione, non è possibile dire quale sia il protocollo più efficace, in 
quantop vari studi hanno utilizzato parametri diversi, giungendo comunque a 
buoni risultati.
La meta-analisi è impedita proprio da questi parametri diversi, dalla scarsa 
numerosità dei pazienti, e dai risultati che vengono forniti solo dei gruppi, e 
non dei singolo pazienti.

Si auspicano quindi ricerche più rigorose, con parametri standardizzati, che 
selezionino pazienti con lesioni uniformi.

\paragraph{Troche} \label{par:tro} \cite{Troche2017} svolge un'analisi di 
alcuni studi per la riabilitazione della disfagia nelle malattie 
neurodegenerative.
Analizza vari paradigmi di riabilitazione, in varie malattie, come Parkinson, 
Scelrosi multipla, ed altre.

Per quanto riguarda la stimolazione elettrica, hanno esaminato tre studi, di 
cui due su PD, con stimolazione cutanea, e uno sulla sclerosi multipla, e 
stimolazione intraluminale.
I due studi su PD non hanno trovato differenze significative tra i gruppi 
sperimentali e di controllo, mentre quello sulla MS ha trovato miglioramenti in 
tutti i parametri considerati.

\paragraph{Di Pede} \ref{par:dip} \cite{DiPede2015} parla anche della 
stimolazione periferica e cerebrale.

La stimolazione periferica porta alla riorganizzazione della rappresentazione 
cerebrale della deglutizione, migliorando gli piutcome deglutitori, ma non è 
chiaro in che misura lo faccia.
Mancano inoltre studi standardizzati, e l'impiego di protocolli diversi rende 
difficile definire l'efficacia.

La stimolazione magnetica cerebrale sembra promettente, anche se ancora poco 
studiata, mentre quella elettrica ha vari vantaggi, come il costo contenuto, la 
possibilità di essere somministrata a letto e la poca invasività, ma altri 
studi sono richiesti per definirne l'efficacia.

Al momento si consiglia questo approccio in associazione con la terapia 
tradizionale.

Le linee guida italiane per la disfagia si concentrano sullo screening e sulla 
valutazione, ma poco sulla riabilitazione.

\paragraph{Steele}\cite{Steele2007} \label{par:ste}  inizia descrivendo i vari 
tipi di stimolatori, e il meccanismo di funzionamento.
Gli stimolatori sono transcutanei, con gli elettrodi appoggiati alla pelle, 
percutanei, con fili sottili che entrano nei muscoli, o impiantati (chiedere se 
il pacemaker può appartenere agli impiantati).

La stimolazione all'interno della cavità buccale e della faringe è stata 
somministrata in vari studi, sia sui pilastri delle fauci bilateralmente, sia 
unilateralmente, sia con degli elettrodi che pendevano all'interno della 
faringe e somministravano la stimolazione durante la deglutizione.
Si è notato che alcune frequenze di stimolazione risultavano positive, con 
tempi di deglutizione e indicatori di disfagia ridotti, mentre frequenze 
maggiori di 5 Hz tendono ad aumentare il tempo necessario per l'inizio della 
deglutizione.
Gli effetti continuano anche dopo la stimolazione, facendo pensare a un qualche 
tipo di riorganizzazione centrale.
Gli studi sono comunque molto contenuti, e alcuni sono stati eseguiti su 
persone sane, limitandone quindi la significatività.

Vengono quindi citati due studi che trattano la stimolazione transcutanea, uno 
con una stimolazione di un'ora al giorno, e uno che invece inizia la 
stimolazione in seguito alla contrazione muscolare che dà inizio alla 
deglutizione.
Entrambi gli studi hanno problemi metodologici che rendono difficile valutare 
l'effettiva efficacia dei metodi utilizzati.

La stimolazione percutanea è stata provata in almeno uno studio, con la 
stimolazione che viene iniziata su comando del soggetto, che preme un bottone 
al momento correto.
Viene riportato che i soggetti sono effettivamente in grado di singìcronizzare 
le azioni, perlomeno i sani, ma proprio il fatto che i soggetti fossero sani, e 
che fossero pochi, rende difficile valutare l'efficacia di questa metodica in 
soggetti disfagici con diverze eziologie.

La conclusione è che non si può ancora essere sicuri dell'efficacia di queste 
metodiche, e che l'uso andrebbe limitato finché non emergano risultati più 
convincenti, da studi di maggiore qualità.

\paragraph{Chen} \label{par:che} \cite{Chen2015} è una meta analisi 
sull'utilizzo della stimolazione elettrica transcutanea nel trattamento della 
disfagia post stroke, con l'obiettivo di definire se la stimolazione elettrica 
sia un'aggiunta valida, con effetti significativi, alla terapia tradizionale, e 
se sia una terapia con effetti significativi anche presa singolarmente.
Per rispondere a queste domande sono stati presi in considerazione solo studi 
randomizzati e quasi-randomizzati.

Nella meta-analisi è stata identificata, in sei studi, una differenza 
significativa tra l'utilizzo e il non utulizzo della stimolazione elettrica, 
con, però, una significativa eterogeneicità.
Dopo la rimozione di uno studio, l'eterogeicità è scomparsa, ma la differenza è 
rimasta significativa, e in favore del ramo sperimentale.
Inoltre, non sono state trovate differenze significative nell'efficacia del 
trattamento neuromuscolare tra i gruppi con stroke acuti e cronici.

Il confronto tra la terapia tradizionale e l'elettrostimolazione senza terapia 
tradizionale non ha mostrato differenze significative, con una bassa 
eterogeneicità.

Gli studi analizzati sono comunque pochi, e l'eterogeneicità emersa potrebbe 
essere causata dalle diverse misure di outcome considerate nei diversi studi.
La meta analisi si concentra solo sui risultati a breve termine, non 
evidenziando eventuali risultati o mancanza di risultati a lungo termine.

\paragraph{Tan} \label{par:tan} \cite{Tan2013} è una meta analisi che confronta 
l'efficacia della terapia tradizionale con quella della stimolazione elettrica 
neuro-muscolare.
Perché un trattamento innovativo possa essere incluso nella cura di una 
condizione, questo deve avere almeno lo stesso livello di efficacia del 
trattamento tradizionale.
Per avere una chiara dimostrazione di efficacia, servono studi 
metodologicamente validi, e con un alto livello di evidenza.

La mata analisi ha trovato che l'uso della stimolazione elettrica transcutanea 
ha esiti significativamente superiori rispetto alla terapia tradizionale.
Questo risultato, però, includeva una eterogeneicità significativa, che è stata 
annullata escludendo uno studio con delle falle metodologiche e risultati molto 
superiori agli altri.

Eseguendo un'analisi dei sottogruppi, si sono accorti che nei pazienti con 
stroke la NMES non risulta significativamente superiore alla terapia 
tradizionale, mentre nei pazienti in cui la disfagia ha altre eziologie la 
differenza è significativa.
Gli autori imputano questa differenza al fatto che nello stroke il danno è 
centrale, e il SNC non riesce a dare lo stimolo necessario per iniziare la 
contrazione in modo corretto e poter sfruttare la NMES.
Nelle patologie come il cancro alla testa e al collo, o i danni da radiazioni, 
invece, la NMES anche da sola dà risultati superiori.

Gli autori evidenziano alcune debolezze dello studio, come lo scarso numero di 
trial presi in considerazione o la presenza di possibili BIAS in questi.

Affermano quindi che la NMES anrebbe inclusa nei protocolli riabilitativi, 
accostata alla terapia tradizionale nei pazienti con stroke, ma in particolare 
andrebbe inclusa nei pazienti disfagici con altra eziologia.
Concludono augurandosi studi futuri di più alta qualità.

\paragraph{Cabib} \label{par:cab:stim} \ref{par:cab} inizia parlando della 
stimolazione elettrica faringea, e di come sembra che la stimolazione 
diminuisca i tempi di transito e l'aspirazione, aumentando anche l'eccitabilità 
cortico-bulbare delle zone coinvolte nella deglutizione.
Questo porta a una diminuzione dell'ospedalizzazione e ad un miglioramento 
della disfagia.

La stimolazione transcutanea, invece, si dimostra efficace nei pazienti 
cronici, specialmente con stimoli sensoriali, anziché motori.
L'efficacia degli stimoli motori è ancora in discussione.
Entrambi i tipi di stimolazione, comunque, sembrano essere sicuri, senza 
particolari eventi avversi che si presentino nei pazienti.
La stimolazione con gli elettrodi posizionati sotto il mento sta dando dei 
buoni risultati in studi su soggetti sani, quindi andrebbe testata anche su 
soggetti patologici.

La stimolazione farmacologica si basa sull'aggiunta di sostanze, come la 
capsaicina, ai boli da deglutire.
Queste aggiunte sembrano diminuire i residui faringei aumentando la forza 
propulsiva della lingua, e inoltre il loro uso sul medio termine promuove la 
neuroplasticità, rendendo più probabile che i vantaggi siano permanenti.

La stimolazione magnetica transcranica viene usata per eccitare (frequenze 
$>1Hz$) o inibire le zone M1, quindi le zone motorie primarie della 
deglutizione.
Si può usare sulla zona colpita dalla lesione, sulla controlaterale, o su 
entrambe.
La stimolazione eccitatoria ipsilaterale alla lesione ha portato ad aumenti 
significativi dell'eccitabilità della zona, e a miglioramenti significativi 
delle scale per la valutazione della disfagia.
La stimolazione eccitatoria bilaterale ha portato a un miglioramento nei 
pazienti con stroke al tronco encefalico.
Anche la stimolazione eccitatoria controlaterale ha dato buoni risultati, 
quando confrontata con la stimolazione sham.
Anche la stimolazione inibitoria controlaterale ha avuto buoni risultati.
Gli autori concludono auspicando una maggiore ricerca sulla stimoalzione delle 
zone sensitive, dato che la ricerca attuale si concentra molto sulle zone 
motorie.

Esiste anche la possibilità, che sta cominciando a venire esplorata, di 
abbinare vari tipi di stimolazione, come la terapia tradizionale e la 
stimolazione transcutanea (per adesso non sembra essere migliore della sola 
terapia tradizionale) o ad esempio la stimolazione magnetica accoppiata con la 
terapia tradizionale.
In tutti questi studi si notano miglioramenti significativi dei pazienti, ma 
servono degli RCT con coorti più estese per poter definire l'efficacia relativa 
di questi protocolli, e la loro fattibilità, anche in rapporto ai costi per la 
sanità.

La stimolazione elettrica transcranica si può somministrare come una 
stimolazione anodica (aumenta l'eccitabilità dell'area su cui viene applicata) 
o catodica (la diminuisce).
Gli studi sono ancora pochi, ma sembra che la stimolazione anodica 
sull'emisfero non affetto porti ad un miglioramento immediato, mentre la 
stimolazione anodica sull'emisfero affetto porti ad un miglioramento della 
disfagia dopo tre mesi dal trattamento.

Alcune meta-analisi recenti mostrano che l'effetto della stimolazione centrale 
è maggiore quando la stimolazione è controlaterale rispetto alla lesione, 
suggerendo un ruolo nella riorganizzazione corticale di queste tecniche.

In sintesi, gli autori affermano che la diagnosi della disfagia sta cambiando, 
dalla semplice analisi dei problemi biomeccanici, all'analisi dell'integrazione 
sensomotoria danneggiata.
Anche il trattamento sta passando dalle semplici strategie compensative alla 
promozione della neuroplasticità, sia per il recupero della deglutizione, sia 
per le funzioni cerebrali.
Per questo, in futuro la riabilitazione potrebbe essere molto diversa da com'è 
oggi, e si rende necessario un cambiamento anche nell'istruzione per stare al 
passo coi tempi.