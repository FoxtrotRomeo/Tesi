\chapter{Diagnosi}
\paragraph{Wirth}\label{par:wir} \cite{Wirth2016} afferma che la disfagia sia 
un problema in crescita nelle persone anziane, soprattutto in quelle 
istituzionalizzate, quindi inizia parlando dei meccanismi neurali alla base 
della deglutizione (controllare references 7 e 8 per la lateralizzazione).

Oltre che negli anziani, la disfagia si presenta anche in altre condizioni, 
neurologiche e non.
Lo stroke ha permesso di studiare la neuroplasticità che permette il recupero 
della deglutizione dopo una lesione, soprattutto grazie al recupero e alla 
compensazione dell'emisfero non affetto.
Le malattie neurodegenerative portano a danni irreparabili, mentre nelle 
malattie a più lenta progressione, come il parkinson, vi è una riorganizzazione 
che limita i danni, fino al raggiungimento di una soglia in cui non è più 
possibile, e il danno si mostra.

Anche condizioni come i tumori della laringe possono portare a disfagia, ma in 
questo caso il danno, più che neuronale, è alle strutture anatomiche 
fisicamente deputate alla deglutizione.

L'apertura esofagea superiore è spesso responsabile della disfagia negli 
anziani: la sezione di questa apertura si riduce, impedendo il corretto 
passaggio del cibo.
Questo viene identificato alla manometria con una pressione del bolo troppo 
alta (dovuta proprio al passaggio impedito), e per un rilassamento incompleto 
dell'apertura.
\'E difficile distinguere le cause alla base di questo problema, e quindi è 
difficile applicare esercizi mirati, quindi conviene iniziare con i meno 
invasivi e proseguire.

Negli anziani, oltre alle malattie, anche i cambiamenti fisiologici collegati 
all'invecchiamento possono influenzare la disfagia, come la riduzione 
sensoriale o quella della forza muscolare.
Anche gli effetti combinati dei farmaci possono essere una causa scatenante 
della disfagia.

Passa poi a parlare delle possibili conseguenze della disfagia, facendo notare 
che per i pazienti le conseguenze psicologiche, come paura, imbarazzo e 
depressione, sono anche più importanti di quelle fisiche.
Nelle conseguenze fisiche vengono indicate: polmonite ab ingestis, che porta ad 
un rischio di mortalità aumentato a 30 giorni dalla diagnosi, a un maggiore 
ricovero ospedaliero e ad un tempo aumentato prima di raggiungere la stabilità 
clinica, e che spesso viene diagnosticata solo come polmonite, dato che la 
disfagia passa sottostimata;
un'ulteriore conseguenza della disfagia è la malnutrizione, e la disidratazione 
del paziente.
Questi fattori influenzano la vita dei pazienti anziani, in particolare di 
quelli istituzionalizzati, e possono portare ad un auto-rinforzo della 
disfagia, dando il via ad un circolo vizioso o spingendo il paziente oltre la 
soglia di fragilità.

La diagnosi viene fatta sulla base di uno screening, di un'indagine clinica, 
e/o sulla base di test strumentali.

Lo screening ha lo scopo di definire le persone a rischio di disfagia, per 
inviarle da un esperto per un test clinico.
Per questo, lo screening deve avere un'alta sensibilità, per non escludere 
persone potenzialmente a rischio.
In aggiunta, deve essere facilmente somministrabile anche da non esperti, in 
modo da includere più persone possibili.

Il test clinico deve invece essere molto specifico, per permettere di creare 
interventi su misura per coloro che sono stati identificati a rischio.
Per questo, deve essere somministrato da un esperto e deve essere molto 
specifico, per non sprec<re risorse con persone che non ne hanno bisogno.

I test strumentali vengono somministrati quando il test clinico non permette di 
avere certezza, o quando non permette di capire quale sia la causa della 
disfagia.

La videofluoroscopia permette di tracciare il percorso di un bolo arricchito 
con un mezzo di contrasto lungo la sua discesa, variando la consistenza e la 
dimensione del bolo nelle varie prove.
Permette quindi di vedere eventuali aspirazioni o penetrazioni del bolo, e 
anche eventuali ristagni.
\'E l'esame gold standard per identificare la disfagia e le strutture che la 
causano, ma a causa della presenza del materiale di contrasto, il suo uso deve 
essere limitato.
Permette di valutare i pazienti e di stabilire l'intervento migliore per loro.

La laringofaringoscopia, invece, è complementare, e permette di studiare 
l'anatomia e la deglutizione durante atti deglutitori eseguiti con del cibo 
vero, in modo da poter ripetere l'esame molto volte (non vi sono radiazioni) e 
da poter studiare, anche con il paziente e i famigliari, come l'adozione di 
determinate strategie influenza la deglutizione.
\'E da notare come non tutti i pazienti che aspirano vadano necessariamente 
incontro a conseguenze nefaste, e che anche le secrezioni possono costituire un 
rischio di aspirazione, oltre che il cibo.

\paragraph{Gonzalez-Fernandez} \label{par:gon} \cite{GonzlezFernndez2015} dà 
una panoramica sui vari tipi di disturbi comunicativi dopo un accidente 
vascolare, evidenziando i substrati anatomici sottesi alla comunicazione e come 
il danneggiamento di ciascuno di questi provochi disturbi diversi (afasia, 
disartria, aprassia del discorso).
Prosegue elencando anche i substrati anatomici sottesi alla deglutizione e 
quali siano le loro funzioni precise, e elenca i tipi di diagnosi ed intervento 
possibili, dicendo su quale parte vanno ad intervenire.

\paragraph{Hines} \label{par:hin} \cite{Hines2010} parla della necessità 
dell'addensamento dei liquidi nei pazienti istituzionalizzati con demenza.
Questa revisione non identifica chiaramente vantaggi nella sopravvivenza di 
questa categoria di pazienti con la modificazione delle consistenze.

I risultati sono che suggerisce l'uso dell'addensante e il monitoraggio da 
parte di un professionista qualificato, e una supervisione nella preparazione, 
perché non sempre lo staff riesce ad essere accurato nella preparazione delle 
consistenze.
Suggerisce inoltre che i pazienti anziani preferiscono l'addensamento dei 
liquidi rispetto alla nutrizione parenterale.

In ultima istanza, mancano studi adeguati per poter definire l'eficacia 
dell'intervento in questa casistica di pazienti.

\paragraph{Di Pede} \label{par:dip} \cite{DiPede2015} è un articolo che esamina 
le strategie riabilitative nella disfagia dell'anziano.

Inizia elencando tutte le condizioni dell'invecchiamento che possono portare a 
disfagia, dividendole in fisiologiche, patologiche e iatrogene.
Le fisiologiche vengono divise a seconda della localizzazione.

Successivamente vengono identificati gli screening validati per la diagnosi 
della disfagia, le valutazioni cliniche e strumentali.
Gli screening validati si dividono in auto amministrati, o a seconda del 
personale che somministra il test.
Molti servono semplicemente per decidere se sia necessario procedere con altre 
analisi, altri permettono anche di definire il grado di severità della disfagia.
Alcuni di questi screening sono specifici per determinate popolazioni di 
pazienti, come i pazienti con stroke o quelli con PD.
Nell'esame clinico, si procede prima senza cibo, poi con il cibo.
Si esamina la funzionalità dei nervi sensitivi (5, 9, 10) e motori (5, 7, 10, 
11, 12) coinvolti nella deglutizione, e le varie fasi (preparazione extraorale, 
fase orale, fase deglutitoria).
La valutazione strumentale è costituita fa faringo-laringo-scopia, 
video-fluoro-scopia, e manometria.
La video-fluoro-scopia è considerata il gold standard, ma alcuni studi 
dimostrano che la faring-laringo-scopia è almeno altrettanto in grado di 
identificare i disturbi della deglutizione, e ha vari vantaggi, ma è bene 
continuare ad utilizzarle entrambe come esami complementari.
La manometria si usa quando c'è una ridotta apertura dello sfintere esofageo 
superiore.

La disfagia nell'anziano dà luogo a numerose complicazioni, come problemi 
nutrizionali, aumentati costi per la sanità, ed altri.

Si passa quindi al trattamento con esercizi.

\paragraph{Edmiaston} \label{par:edm} \cite{Edmiaston2009} spiega il processo 
per creare e validare un nuovo test da fare a bordo letto per lo screening 
della disfagia.
Per iniziare elenca altri 6 test, e il loro processo di validazione, quindi la 
coorte di pazienti selezionati nel crearli, e le misure di efficacia che sono 
state prese in considerazione.
Spiega poi come ciascuno di questi test definisca la condizione di disfagia nei 
pazienti, e ne elenca i difetti, come misure di efficacia non complete, o 
coorte di pazienti utilizzati per la validazione troppo ristretta.

Il test che è stato elaborato in questo studio è stato creato per essere 
somministrato da infermieri ed altri professionisti della salute, quindi deve 
essere composto da item semplici, ma comunque calzanti nel definire la 
disfagia. 

Il test qui valutato è composto da una valutazione dello stato di coscienza, 
dei muscoli faciali, della lingua e del palato, e infine dal 3 oz water test.

Il personale è stato addestrato da un logopedista in 10 minuti per ogni 
persona, e questo ha permesso di avere una buona sensitività, di oltre il 90\%.
Inoltre, i pazienti sono stati valutati più in fretta di quello che avrebbe 
permesso la valutazione da parte del logopedista, in modo da far riprendere 
loro una dieta il più possibile normale in fretta.

I buoni parametri di affidabilità di questo test, e il basso costo in termini 
di tempo determinato dall'addestramento del personale, lo rende un buon test, e 
determina come sia possibile anche per il personale non specializzato svolgere 
un buon lavoro nella valutazione inziale dei pazienti.

\paragraph{Yang} \label{par:yan} \cite{Yang2016} è una review molto corposa 
sull'utilizzo dell'imaging nella diagnosi della disfagia.
In particolare parla dell'imaging correlato alla deglutizione e ai movimenti 
della lingua, ma dà indicazioni solo molto marginali per la riabilitazione.

Per quanto riguarda la riabilitazione, dice che l'input sensoriale dato dalla 
lingua è probabilmente molto importante per iniziare il processo deglutitorio, 
ed è quindi fondamentale includere degli stimoli sensoriali nella 
riabilitazione della deglutizione.
Dice inoltre che, a seconda della gravità del paziente, può essere più o meno 
adeguato intraprendere questo percorso, suggerendolo più nei pazienti con 
problemi moderati.
In ogni caso, è sempre necessaria la presenza di un professionista addestrato 
per la sicurezza del paziente.