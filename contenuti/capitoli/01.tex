\chapter{Ricerca delle fonti}
La ricerca è stata effettuata su PubMed, con il MeSH term \texttt{"Deglutition 
Disorders/rehabilitation"[Mj:noexp]}.
Eliminando l'esplosione si sono esclusi 1600 risultati, permettendo di 
concentrarsi maggiormente sugli articoli centrati sulla riabilitazione, e 
soprattutto escludendo molti articoli che trattavano condizioni non correlate 
con la ricerca, come l'acalasia.

Di questi articoli si sono privilegiate le revisioni, dato che lo scopo della 
tesi è innanzitutto quello di fare una revisione delle attuali linee guida per 
il trattamento della disfagia.

Inserendo quindi la stringa \texttt{"Deglutition Disorders/rehabilitation"}\\
\texttt{[Mj:noexp] AND Review[ptyp]} per analizzare le revisioni, si sono 
individuati 
in totale 60 articoli, di cui si sono letti i titoli.
36 di questi 60 articoli sono stati esclusi, perché relativi a condizioni che 
non riguardano la ricerca (come i tumori al collo), a casistiche di pazienti 
che non riguardano la ricerca (come i pazienti pediatrici), o perché precedenti 
alla redazione delle linee guida, quindi al 2007.

Gli articoli precedenti al 2007 sono stati esclusi in quanto dovrebbero già 
essere inclusi nelle linee guida, quindi non apporterebbero cambiamenti a 
quanto affermato dalle linee guida stesse.
Se si rendesse necessario leggere questi articoli perché altri articoli vi 
fanno importanti riferimenti, naturalmente, si procederà alla lettura.

Si passa ora a leggere gli abstract delle 24 revisioni individuate, per meglio 
capire se possano fare al caso nostro o meno.

Dalla lettura degli abstract, emerge che 3 articoli non sono disponibili in 
inglese, e vengono quindi eliminati dalla ricerca.

I restanti articoli possono essere divisi in alcune categorie principali, di 
cui le più rappresentate sono:
\begin{itemize}
	\item Post-stroke oropharingeal dysphagia, con 4 review
	\item Electric stimulation, con 4 review
	\item neural plasticity, con 3 review.
\end{itemize}
Gli altri articoli vertono sull'esercizio, le strategie compensatorie, la 
stimolazione transcranica, la gestione multidisciplinare e l'analisi neurale.

La maggior parte di queste revisioni confronta l'efficacia di un trattamento 
innovativo con un trattamento tradizionale.

Questi articoli saranno analizzati e confrontati con le linee guida italiane 
sul trattamento della disfagia, e con quelle dell'associazione mondiale dei 
gastroenterologi.
Le linee guida italiane sono abbastanza datate (sono del 2007) e per questo si 
è resa necessaria una loro revisione alla luce delle recenti acquisizioni del 
monso scientifico.

In particolare, si cercheranno eventuali differenze tra le linee guida italiane 
e le revisioni di letteratura successive, per poter elaborare un questionario 
volto a verificare la preparazione degli operatori (medici, logopedisti, 
fisioterapisti, infermieri ed operatori socio sanitari) sull'argomento.
Questo questionario verrà somministrato alle figure sopra elencate in contesti 
diversi (case di riposo ed ospedali), in modo da verificare la loro 
preparazione, e da poter operare un confronto tra operatori di strutture 
diverse.

\'E stata eseguita una ricerca in CINAHL Complete with Full Tex, un database di 
articoli infermieristici e per le professioni sanitarie, e in PsychINFO, un 
database di letteratura psicologica e comportamentale.
La stringa inserita è stata \texttt{(MM "Deglutition Disorders/TH/RH") AND (MH 
"Systematic Review")} e si sono individuati 9 risultati.
Un articolo è stato escluso in quanto duplicato di uno già trovato su PubMed, 
ed un altro è stato escluso perché riferito a pazienti pediatrici.

Si è passati alla ricerca sul database dell'American Association of critical 
care nurses, inserendo semplicemente la stringa di ricerca \texttt{Dysphagia}, 
e si sono individuate 5 risorse.
Due di queste risorse erano seminari sulla disfagia dopo l'estubazione, e sono 
stati quindi esclusi.
Altri due erano articoli sulla validazione di un test diagnostico per pazienti 
che fossero stati estubati dopo una lunga intubazione, e sono stati quindi 
esclusi.
Il rimanente articolo conteneva la validazione di un test diagnostico per 
pazienti che avessero subito uno stroke, e che fosse somministrato non da 
logopedisti.
L'articolo è stato incluso nella ricerca, in quanto sembra essere una buona 
risorsa per sondare la preparazione di figure professionali diverse da quelle 
del logopedista rispetto allo screening della disfagia.

\'E stata eseguita anche la ricerca su PEDro, il database evidence-based di 
risultati collegati alla fisioterapia.
Inserendo nella ricerca avanzata le stringe \texttt{Deglutition disorders} o 
\texttt{Deglutition disorder}, in associazione con il tipo di articolo 
Systematic Review, non sono emersi risultati.
Si è quindi provato con una stringa più generica, come \texttt{Dysphagia}, 
facendo emergere 24 risultati, di cui 3 sono stati eliminati perché 
riguardavano pazienti oncologici.
Un ulteriore articolo è stato escluso perché trattava pazienti pediatrici.
6 articoli sono stati eliminati in quanto già trovati in ricerche precedenti.
Un articolo è stato eliminato perché non disponibile in inglese.

\'E stata eseguita una ricerca sulla Cochrane Library, cercando 
\texttt{Deglutition disorder}.
Dalla ricerca sono emerse 13 revisioni, di cui 3 riferite a pazienti 
oncologici, che sono quindi state escluse.
Una ulteriore revisione è stata esclusa perché riguardante pazienti pediatrici.
Altre 3 erano relative a procedure mediche o chirurgiche, e sono quindi state 
escluse.
Una ulteriore revisione è relativa alla pratica dell'agopuntura, e bisogna 
decidere se tutte le revisioni che parlano di questo argomento vadano o meno 
incluse.
Delle revisioni incluse, una ha dovuto essere eliminata perché già inclusa 
dalle precedenti ricerche.

Dai 36 articoli risultanti si è proceduto a controllare le date di 
pubblicazione, eliminando gli articoli già disponibili nel 2007.
Si è quindi proceduto a categorizzare gli articoli con argomenti simili, come 
per esempio l'elettrostimolazione, per procedere ad una più rapida analisi.
Inoltre, trattandosi in gran parte di revisioni sistematiche e meta-analisi, un 
articolo successivo probabilmente includerà tutte le fonti primarie di uno 
precedente, a meno che non si tratti di un update.
Un'analisi eseguita in questo modo permette di snellire notevolmente la mole di 
lavoro.

Un articolo sulla neuriabilitazione è stato escluso perché riportava lo 
svolgimento di una consensus conference, ma senza parlare dei risultati clinici 
o scientifici.
 
