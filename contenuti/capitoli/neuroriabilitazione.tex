\chapter{Neuroplasticità}
\paragraph{Robbins 2008} \label{par:rob2008} \cite{Robbins2008} inizia 
elencando i principi della neuroplasticità, e indicando come ciascuno di questi 
sia collegato ad uno o più aspetti della riabilitazione della deglutizione.

Prosegue elencando i substrati anatomici, e come anche questi vengano 
influenzati dagli esercizi, e i vari tipi di interventi (sensoriale, variazioni 
nelle caratteristiche del bolo, tramite stimolazioni, riabilitatori con e senza 
deglutizione) e come anche questi influenzino la plasticità cerebrale.

Passa poi ad evidenziare le evidenze di plasticità comportamentale e neurale 
collegate alla riabilitazione, e a parlare dei tipi di ricerca auspicabili in 
futuro per espandere le conoscenze in questo campo.

Sono molto interessanti le tabelle in cui si mostrano le correlazioni.

