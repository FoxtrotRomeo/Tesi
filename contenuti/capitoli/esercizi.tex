\chapter{Esercizi}
\paragraph{McKenna} \label{par:mck} \cite{McKenna2017} lingual strenght review.
L’articolo prende in considerazione studi su persone sane e patologiche, 
analizzando la pressione di picco della lingua durante esercizi isometrici.
Gli esercizi vengono divisi in base alla zona di applicazione della pressione, 
e la popolazione viene analizzata a seconda di caratteristiche come sesso, età, 
ed eziologia della patologia nei non sani.

Viene citato uno studio del 2006 in cui si dice che viene data la durata 
raccomandata per i trattamenti della disfagia (Kays \& Robbins, 2006), che 
potrebbe essere da leggere.

4 dei 5 studi che includono partecipanti con disfagia mostrano che la pressione 
della lingua sul palato all’inizio del trattamento è sotto il livello di norma 
per gli adulti in salute con condizioni simili.
Anche nelle persone in salute, l’allenamento isometrico ha portato a un 
significativo aumento delle misure di pressione. I miglioramenti nelle manovre 
di deglutizione sono meno evidenti.
In casi di disfagia grave, in cui il paziente non è nemmeno in grado di 
iniziare la manovra di deglutizione volontariamente, l’allenamento potrebbe 
effettivamente dare un aiuto.

La review da sola non basta per dire se l’allenamento isometrico della lingua 
permetta di migliorare le prestazioni nella deglutizione: erano troppo diversi 
i protocolli utilizzati, le casistiche di pazienti, e anche i test utilizzati 
per misurare gli outcome.
Sembrano esserci dei buoni risultati, ma non per il cancro al collo e al capo, 
probabilmente dovuti alla complessità della patologia.

\paragraph{Langmore}\label{par:lan} \cite{Langmore2015} è una revisione non 
sistematica della 
bibliografia.
Include studi che mostrino gli esercizi presi singolarmente, in modo da isolare 
l'effetto del singolo esercizio, piuttosto che della combinazione.

Fa un recap dei meccanismi della neuroplasticità che sono alla base del 
recupero nella disfagia, come \textit{Use it or lose it}, \textit{Use it and 
improve it}, specificità, trasferibilità e intensità per gli esercizi 
deglutitori, e trasferibilità e intensità per i non deglutitori.

Divide gli esercizi in deglutitori e non deglutitori.
I deglutitori includono la deglutizione impegnativa \textit{(effortful 
swallowing)}, la manovra di Masako, il protocollo di McNeill, la manovra di 
Mendelsohn e la sopra-sovraglottica.
I non deglutitori includono il sollevamento della testa di Shaker, il rinforzo 
della lingua, il trattamento della voce di Lee Silverman e l'allenemento dei 
muscoli espiratori.

In tutti questi esercizi ci sono stati problemi con molti studi, che erano 
effettuati sui sani, o senza coorti abbastanza estese o con un follow-up troppo 
breve.
La conclusione della revisione, quindi, è che prima di tutto servono ulteriori 
studi di alta qualità nel campo, quindi che la manovra di Mendelsohn può essere 
consigliata (con cautela) per le evidenze limitate, mentre per le altre manovra 
deglutitorie non ci sono evidenze sufficienti.
Tra le menovre non deglutitorie, si possono consigliare il sollevamento della 
testa di Shaker e l'allenamento dei muscoli espiratori (per i pazienti con PD), 
mentre per le altre manovre non ci sono ancora evidenze sufficienti.

\paragraph{Antunes} \cite{Antunes2012} è una revisione sistematica per gli 
effetti del sollevamento della testa di Shaker nei pazienti anziani.
Analizza 9 studi, di cui alcuni vertono sull'esercizio somministrato a soggetti 
sani, altri su patologici.
Alcuni di questi erano su anziani, e altri su giovani adulti.
I vari studi hanno disegni diversi, ma concordano nell'affermare che produca 
dei cambiamenti nel diametro dell'apertura esofagea, e nell'escursione della 
laringe.
Solo pochi studi esaminano le differenze sulle capacità deglutitorie dei 
soggetti.

Il problema principale della revisione è l'eterogeneità dei disegni degli 
studi, assieme alla scarsa numerosità dei soggetti.
Questo non permette di dare indicazioni cliniche di alta qualità, anche se i 
risultati provvisori sono concordi in tutti gli studi, e si auspicano studi 
successivi su pazienti con stroke.

\paragraph{Crary}\label{par:cra} \cite{Crary2014} analizza 4 diversi esercizi 
per la disfagia, tutti considerati innovativi, o che inseriscano innovazione 
nel modo di vedere l'esercizio.
Si riferisce che tutti questi esercizi hanno prodotto cambiamenti nel 
meccanismo della deglutizione, o negli outcome della deglutizione, ma che 
purtroppo hanno bisogno di ulteriori indagini rigorose, per poter quantificare 
il beneficio e indirizzarlo a coorti specifiche.

Il lavoro sul rinforzo della lingua funziona, ma non con pazienti che abbiano 
il cancro al collo e alla testa, come diceva anche McKenna \ref{par:mck} 
probabilmente a causa della situazione clinica particolarmente complicata di 
questi pazienti, che rende necessario un intervento più specifico e strutturato.
Il lavoro sul sollevamento della testa sfrutta bene il principio della 
resistenza progressiva, grazie alla fatica a cui va incontro la muscolatura del 
collo col progredire dell'esercizio.
Viene citata \cite{Antunes2012} come prova del fatto che questo esercizio sia 
promettente.
Il protocollo di McNeill è un programma sistematico che utilizza la 
deglutizione di volumi e materiali diversi come esercizio.
Si è dimostrato più efficace della terapia tradizionale per migliorare la 
deglutizione, il contatto della lingua col palato, l'escursione della laringe, 
il timing di deglutizione e le pressioni della faringe.

\paragraph{Robbins 2013}\label{par:rob2013}\cite{Robbins2013} elenca le 
complicazioni 
della disfagia, come possibilità di polmonite \textit{ab ingestis} e di 
malnutrizione e disidratazione, i possibili test di screening, i fattori di 
rischio, come età, livello cognitivo, localizzazione e gravità dello stroke e 
comorbidità.

Si evidenzia una neuroplasticità dei circuiti neuronali deputati alla 
deglutizione, che rende particolarmente importante l'intervento, come anche il 
fatto che senza intervento precoce si rischia di avere l'atrofia de muscoli.

Dà inoltre indicazioni sulla gestione e sul trattamento.
Nella gestione si parla di nutrizione non orale, che non previene l'aspirazione 
delle secrezioni orali o dei reflussi gastrici, ma che spesso è comunque 
necessaria per brevi periodi.
La PEG viene identificata come più sicura e con minori outcome di bassa qualità 
di vita rispetto al sondino naso-gastrico.
\'E importante discutere a fondo l'argomento per la nutrizione alternativa con 
il paziente, per concordare assieme gli obiettivi.

Nell'intervento, le tecniche compensatorie si dividono in: posture, 
modificazioni alla dieta (l'eccessivo addensamento dei liquidi (consistenza 
miele) sembra portare ad una maggiore incidenza di polmoniti ab ingestis, in 
quanto se avviene l'aspirazione è più difficile ottenere una clearance dei 
polmoni), igiene orale e strategie nel mangiare (adattamento 
ambientale).
L'intervento indiretto si divide in: alterazioni del bolo (studi di scarsa 
qualità e l'effetto potrebbe finire presto dopo la stimolazione), stimolazione 
orale termo-tattile (non chiaro se sia più importante il termo o il tattile, e 
non chiaro l'effetto sull'aspirazione), applicazione di pulsazioni (non chiaro 
l'effetto correlato con l'intensità, il timing o le ripetizioni, come anche la 
localizzazione.), allenamento senza deglutizione e allenamento della lingua.
L'intervento diretto si divide in: manovra di Mendelsohn, manovra 
sovra-glottica e super sovra-glottica (che potrebbe essere da sconsigliare nei 
pazienti con stroke, perché potrebbe portare a problemi circolatori, secondo 
una revisione), biofeedback.

In sintesi, si evidenzia l'importanza di un'intervento intensivo e precoce, e 
del team multidisciplinare.
Si afferma che l'intervento intensivo permette di migliorare il ritorno ad una 
dieta normale a sei mesi dall'ictus, e che riduce il rischio di 
ri-ospedalizzazione in seguito a polmoniti ab ingestis.

\paragraph{Wirth}\ref{par:wir} Parla anche degli esercizi disponibili, 
distinguendoli in tecniche compensative ed esercizi riabilitativi.
Spiega come alcune di queste tecniche siano molto efficaci, sia per ridurre 
immediatamente il rischio di aspirazione, sia a lungo termine per modificare la 
fisiologia della deglutizione, mentre gli esercizi non sempre sono efficaci, o 
non sempre possono essere proposti ad una popolazione anziana.

Evidenzia l'importanza di studi seri per definire l'efficacia degli esercizi in 
determinate coorti di pazienti, con criteri rigorosi.

La modificazione dei cibi viene indicata come importante per migliorare la 
sicurezza della deglutizione, ma si evidenzia come non vi sia uniformità nel 
definire le consistenze, e come questi cibi siano poco appetitosi per i 
pazienti, portanto a un calo di appetito.
Mancano inoltre studi sufficienti per definire se effettivamente il cambio di 
consistenze sia un intervento efficace, e in quali casi si debba adottare quale 
consistenza.

\paragraph{Troche}\ref{par:tro} continua con l'esame di esercizi sulla forza 
dei muscoli respiratori.
Questi esercizi si eseguono grazie ad oggetti che permettono di modificare la 
pressione o la resistenza applicata agli atti respiratori.
In questi studi i pazienti hanno PD, Corea di Huntington, o SLA.

Nei pazienti con PD e una disfagia moderata, si è trovato che 4 settimane, 5 
giorni a settimana, 5 serie da 5 ripetizioni al giorno hanno portato a 
miglioramenti significativi nella scala di penetrazione aspirazione, mentre i 
controlli non ne hanno avuti.
Dopo tre mesi di decondizionamento non ci sono stati peggioramenti in questi 
parametri, ma la severità della disfagia non era uniforme, e questo potrebbe 
aver reso poco chiari i risultati.
In un altro studio si sono trovati anche miglioramenti significativi per 
l'efficacia della tosse.

Per quanto riguarda la Corea, il trattamento è durato 4 mesi, e sono stati 
trovati grandi miglioramenti nei parametri respiratori, ma non in quelli della 
deglutizione o della SWA-QoL.

Con la SLA il trattamento è durato 5 settimane, trovando grandi miglioramenti 
nei valori della respirazione, ma non in quelli della deglutizione, anche se il 
trattamento è stato identificato come fattibile e sicuro per questi pazienti.

Prosegue con l'esame del buofeedback come strumento in pazienti con PD.
Il biofeedback consiste nel dare una misura al paziente di come sta eseguendo 
l'esercizio, tramite videolaringoscopia o tramite elettromiografia.
Nel primo caso, si sono osservati miglioramenti significativi nel residuo 
post-deglutitorio, nel secondo sui tempi e volumi di deglutizione.
In entrambi i casi ci sono stati miglioramenti significativi sulla qualità 
della vita.

Continua con il Lee Silverman Voice Treatment, riportando uno studio su una 
coorte di 8 pazienti con PD, che dopo il trattamento hanno mostrato un 
miglioramento dei tempi di transito per certe consistenze, e dell'efficienza 
della deglutizione.

Controlla con un protocollo di esercizi per la mobilità orale (fonazione 
sostenuta, elevazione della laringe, retrazione della lingua, succhio, e 
respiro trattenuto) in pazienti con PD.
Ci sono stati miglioramenti significativi per quanto riguarda il residuo, il 
controllo del bolo, e la qualità della vita.

In conclusione, questi studi riportano che anche nelle malattie 
neurodegenerative gli esercizi sono sicuri e fattibili, mentre non riportano 
spesso risultati conclusivi, o non confrontano l'efficacia del variare di 
alcuni parametri, anche per la scarsa numerosità dei soggetti di studio.
Studi successivi, di alta qualità, che investighino soggetti con parametri 
diversi sono richiesti per poter creare una pratica clinica veramente basata 
sulle evidenze.

\paragraph{Deane} \label{par:dea} \cite{Deane2001} è una review con contenuti 
aggiornati al 2001, ma pubblicata nel 2009.

Riferisce di aver cercato, senza successo, degli RCT in cui si comparasse 
l'efficacia di trattamenti non farmacologici per la disfagia con trattamenti 
placebo o con nessun trattamento, in pazienti con PD.

All'inizio della review si evidenzia come fosse molto bassa, negli anni 
passati, la percentuale di persone con PD che venivano visitate ed 
eventualmente trattate da logopedisti.
Per questo, la presenza di un fisioterapista (che invece è più presente, per 
trattare i disturbi del movimento) che sia anche istruito nell'individuazione e 
nella cura della disfagia potrebbe fare una grande differenza nella diagnosi 
precoce, nella qualità di vita e nei costi per il sistema sanitario.

Prosegue indicando le caratteristiche che dovrebbero avere gli studi che si 
propongano di analizzare l'efficacia di questa terapia in questi pazienti.

\paragraph{Di Pede} \ref{par:dip} \cite{DiPede2015} passa a parlare della 
riabilitazione, dividendo gli interventi in compensatori e riabilitativi.

Gli interventi compensatori sono ad esempio quelli posturali, come mangiare con 
la schiena verticale, portare il mento verso il petto, o, nei pazienti 
emiparetici girare la testa dal lato paretico, per chiudere l'accesso alle vie 
respiratorie da quel lato.
Bisogna fare attenzione anche alla quantità di cibo, per evitare la 
denutrizione, e al modo in cui si mangia, per prevenire i problemi, ad esempio 
evitando le consistenze multiple.
Le consistenze vengono modificate a seconda del problema del paziente, 
addensando i liquidi o frullando i solidi, e l'igiene deve essere ben fatta per 
ridurre le infezioni.

Negli interventi riabilitativi si cerca di rinforzare muscoli specifici o 
gruppi muscolari.
Nella deglutizione forzata si cerca di migliorare lo spostamento della base 
della lingua, e l'escursione dell'osso ioide, con una deglutizione in cui il 
paziente deve impegnarsi particolarmente nell'accentuare i movimenti.
Si sono notati miglioramenti visibili, ma in coorti di pazienti sani.
Nella manovra super- o sovra-glottica, la persona deve trattenere il fiato, 
deglutire, e quindi tossire o raschiare la gola.
Con questa manovra si nota che le vie aeree restano chiuse più a lungo e ci 
sono meno residui, ed è consigliata per i pazienti con chiusura delle vie aeree 
ridotta, escursione ridotta della lingua o della laringe, ma sconsigliata in 
chi ha problemi pressori.
La manovra di Masako si fa deglutendo mentre la punta della lingua viene 
trattenuta tra i denti, per rinforzare la base della lingua e aumentare il 
movimento del muro posteriore della faringe.
Si esegue sono con la saliva, in quanto aumenta l'apertura delle vie aeree e il 
tempo di deglutizione.
La manovra di Mendelsohn consiste nell'aumentare l'escursione faringea 
mantenendo la faringe in posizione con le dita durante un atto deglutitorio.
L'esercizio di Shaker mira ad aumentare la forza dei muscoli sopraioidei 
tramite delle alzate della testa, per aumentare l'escursione della laringe.

Il trattamento di McNeill utilizza l'atto deglutitorio come allenamento, quindi 
ha delle basi migliori dal punto di vista funzionale e della neuroplasticità, 
ed è risultato superiore al trattamento tradizionale in almeno due studi.

L'iniezione di botulino permette di trattare le disfunzioni dello sfintere 
esofageo superiore, ma deve essere somministrata da medici esperti.

I trattamenti farmacologici differiscono a seconda del problema che porta 
all'origine della disfagia, vengono usati molti miorilassanti per ridurre gli 
spasmi della muscolatura liscia nei problemi esofagei distali, ma esistono 
anche altri approcci, se il problema è differente.

Si passa quindi alla stimolazione.

\paragraph{Vogel} \label{par:vog} \cite{Vogel2015} è una revisione Cochrane che 
si proponeva di trovare le evidenze per il miglior trattamento nella disfagia 
per i pazienti affetti da atassia ereditaria.
Le atassie ereditarie sono un gruppo di malattie a trasmissione genetica, che 
possono essere dominanti o recessive, che porta a rpogressiva incoordinazione e 
problemi progressivi nella visione, nel parlare, nelgli aspetti cognitivi e 
nella deglutizione.
L'incidenza di ciascuna di queste malattie è molto bassa, ma dato che sono 
molte, diventano più frequenti come gruppo.

L'incidenza della disfagia nei pazienti con atassia ereditaria è molto alta, ed 
aumenta con il progredire della malattia.

Gli autori cercavano RCT e q-RCT specifici per pazienti con atassia ereditaria, 
in modo da poter avere delle buone evidenze sull'efficacia del trattamento in 
queste malattie.
Non sono risultati articoli includibili nella ricerca.

Per questo, gli autori concludono che mancano ancora delle buone evidenze 
sull'efficacia del trattamento per la disfagia in queste malattie, e che 
servono ulteriori studi, condotti con una buona metodologia, e probabilmente 
multicentrici ed a livello internazionale per supplire alla scarsità di 
pazienti.
Il trattamento che viene messo in atto per ora è un trattamento aspecifico, 
composto di modificazioni ambientali e comportamentali, di mofificazioni 
posturali, modificazioni alla consistenza dei cibi, ed esercizi come 
l'effortful swallowing.
Non ci sono, quindi, raccomandazioni specifiche per l'atassia ereditaria.

Gli autori notano che anche altri studi Cochrane nel campo delle malattie 
neurologiche degenerative hanno trovato evidenze insufficienti per poter trarre 
conclusioni sull'efficacia, o sull'efficacia relativa, di trattamenti 
specifici. 

\paragraph{Cabib} \label{par:cab} \cite{Cabib2016} è un articolo molto lungo e 
corposo, con un bellissimo schema iniziale sui vari problemi correlati alla 
disfagia.

Parla di strategie compensatorie, come l'addensamento degli alimenti (alcuni 
addensanti possono portare a un maggior residuo faringeo), dell'abbassamento 
del mento (solo la metà dei pazienti ne trae beneficio, in particolare quelli 
che hanno disfagia solo con grandi volumi) e dell'arricchimento sensoriale, che 
sembra promettente e sembra promuovere la neuroplasticità.
Tutti gli studi citati fino a questo punto non vengono spiegati in dettaglio, 
rendendo quindi difficile pensare che possano essere di buona qualità.

Parla poi di stimolazione elettrica, magnetica e farmacologica. 
\ref{par:cab:stim}

\paragraph{Smith} \label{par:smi} \cite{Smith2012} si occupa di definire se sia 
meglio un approccio compensatorio o uno riabilitativo nella disfagia per 
panzienti con PD.

Un approccio compensatorio è un approccio che cerca di migliorare la sicurezza 
e la facilità del mangiare e del bere, senza effettivamente cambire la 
fisiologia della deglutizione.
Un approccio riabilitativo, invece, migliora la funzione deglutitiva 
cambiandone effettivamente la fisiologia, con esercizi motori o di integrazione 
senso-motoria.

Gli approcci compensativi considerati in questa revisione sono principalmente 
l'utilizzo di addensanti per i liquidi e l'abbassamento del mento durante la 
deglutizione.
Sull'utilizzo degli addensanti alcuni studi sono discordi, e non si riesce 
quindi a definire se siano effettivamente meglio delle posture.
L'utilizzo di addensanti con consistenze molto dense, però, sembra produrre 
disagio nei pazienti.
L'utilizzo delle posture in alcuni studi risulta inferiore a quello degli 
addensanti.
Inoltre, questi interventi non risolvono il problema del paziente nel 
medio-lungo periodo, e devono essere adottati costantemente per evitare 
complicazioni, portando comunque ad un abbassamento della qualità di vita del 
paziente.

Gli approcci riabilitativi considerati sono stati: deglutizione forzata in 
combinazione con l'uso di biofeedback, Lee Silverman Voice Treatment, esercizio 
dei muscoli espiratori.
In tutti i casi si sono notati dei miglioramenti in alcuni parametri, come il 
ritardo nell'inizio della deglutizione, o i residui faringei con certe 
consistenze.
La qualità degli studi, però, è risultata solitamente bassa, con vari bias 
possibili e con delle coorti molto ristrette.
Per questo, anche se i risultati sono definiti incoraggianti, gli autori 
auspicano maggiori ricerche in questo campo, e un cambiamento di approccio da 
parte dei clinici.
Gli autori dicono, infatti, che l'utilizzo di esercizi riabilitativi, per 
quanto possa avere dei risultati meno immediati nel tempo, può portare a 
risultati più duraturi, ad una minore spesa sanitaria, e a un miglioramento 
della qualità della vita dei pazienti, che non sarebbero più obbligati a 
concentrarsi sulla postura o ad assumere consistenze modificate ad ogni pasto.

\paragraph{Regan} \label{par:reg} \cite{Regan2014} è una revisione Cochrane 
sull'impiego della tossina botulinica nel trattamento delle disfunzioni dello 
sfintere esofageo superiore.
Queste disfunzioni possono provocare disfagia e aspirazione, perché se il cibo 
o i liquidi non riescono a passare agevolmente lo sfintere esofageo si trovano 
a restare al disopra, quindi vicino all'ingresso delle vie aeree, e possono 
essere facilmente inalati.
inoltre, una disfunzione di questa struttura aumenta facilmente la quantità di 
residui che restano nella faringe dopo la deglutizione.

Quando la riabilitazione non funziona (esercizi di sollevamento della testa di 
Shaker, esercizi per la mascella, esercizio di Mendelsohn) e i comportamenti 
compensativi (abbassamento del mento o rotazione della testa durante la 
deglutizione) non sono sufficienti, si passa all'intervento chirurgico o 
farmacologico.
L'intervento farmacologico, con l'iniezione della tossina botulinica nella 
muscolatura cricofaringea, può avvenire in vari modi.
Sono possibili effetti indesiderato, ad esempio, in seguito all'iniezione nel 
punto sbagliato si possono paralizzare dei muscoli sbagliati.
Per questo, si cercavano degli studi RCT per definire quali fossero gli 
interventi più sicuri ed efficaci.

Non si sono trovati RCT sull'argomento.
La revisione conclude quindi auspicando la produzione di studi adeguati e 
metodologicamente corretti per produrre delle buone evidenze e trattare i 
pazienti nel modo migliore possibile.