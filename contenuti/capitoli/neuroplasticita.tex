\chapter{Neuroplasticità}
\paragraph{Robbins 2008} \label{par:rob2008} \cite{Robbins2008} inizia 
elencando i principi della neuroplasticità, e indicando come ciascuno di questi 
sia collegato ad uno o più aspetti della riabilitazione della deglutizione.

Prosegue elencando i substrati anatomici, e come anche questi vengano 
influenzati dagli esercizi, e i vari tipi di interventi (sensoriale, variazioni 
nelle caratteristiche del bolo, tramite stimolazioni, riabilitatori con e senza 
deglutizione) e come anche questi influenzino la plasticità cerebrale.

Passa poi ad evidenziare le evidenze di plasticità comportamentale e neurale 
collegate alla riabilitazione, e a parlare dei tipi di ricerca auspicabili in 
futuro per espandere le conoscenze in questo campo.

Sono molto interessanti le tabelle in cui si mostrano le correlazioni.

\paragraph{Martin} \label{par:mar} \cite{Martin2009} indica che la 
neuroplasticità può essere determinata da vari tipi di interventi esterni, o di 
altri fattori.
Gli interventi esterni sono, ad esempio, comportamentali (in cui si chiede al 
paziente un certo comportamento durante l'intervento) o non comportamentali (in 
cui il paziente è essenzialmente passivo durante l'intervento).

I non comportamentali sono essenzialmente stimolazioni elettriche periferiche 
(a seconda della frequenza e della localizzazione possono aumentare o diminuire 
l'eccitabilità della corteccia motoria faringea a vari intervalli di tempo 
dalla stimolazione), stimolazioni non elettriche (meccaniche, termiche, 
gustative, deafferenziazione) periferiche (a seconda dello stimolo aumentano o 
diminuiscono l'eccitabilità, e stimolazioni magnetiche transcraniche (a seconda 
dell'emisfero stimolato e della frequenza aumentano o diminuiscono 
l'eccitabilità e l'interferenza interemisferica).

I comportamentali sono invece quelli che richiedono un allenamento degli 
effettori del paziente, quindi allenamento della lingua, ma non solo, anche 
altri protocolli, che però non sempre hanno esaminato l'effetto sulla 
neuroplasticità degli esercizi.
Gli esercizi per la lingua portano ad un aumento della rappresentazione 
corticale della lingua, in accordo con il principio di specificità, e seguono 
altri principi, come la salienza.
Purtroppo non è detto che la deglutizione migliori grazie alla neuroplasticità, 
in questi casi, ma forse solo grazie alla migliore azione della lingua.
Servono quindi ulteriori studi che si rivolgano ad altri tipi di esercizi.

Una buona idea potrebbe essere di sommare interventi comportamentali e non 
comportamentali, ad esempio facendo eseguire esercizi per la lingua o la 
deglutizione durante la stimolazione magnetica transcranica.

Anche altrei eventi esterni, come i traumi centrali o periferici, possono 
influenzare la neuroplasticità, aumentando o diminuendo l'attivazione di 
determinate zone in risposta a determinate azioni.
Anche interventi periferici, come il taglio di un nervo o l'estrazione di un 
dente, modificano in modelli animali la rappresentazione corticale.
Sono tutte cose da tenere in considerazione durante la riabilitazione di un 
paziente.

